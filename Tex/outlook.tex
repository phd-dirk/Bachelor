\chapter{Outlook}
	The band structure and density of states comparisons in this thesis contribute to the continuing interest in analyzing the unique electronic properties of graphene. \\
	Graphene has some very special characteristics in heat conductivity, thinness, linear energy dispersion (in the vicinity of the Dirac point) and electric conductivity being a zero gap semiconductor. But what are the field's unsolved problems and future challenges? \\\\
	Within the framework of the present thesis we need to adjust the FPLO basis to avoid negative density of states. Furthermore we should regard other compounds as BC$_3$ mono-lattice, which already has been synthesized in \cite{BC3}. Beside doping graphene we also should create sandwich sandwich structures. For example one layer graphene followed by an layer consisting of another element or doped graphene.\\
	Regarding our tight binding model we need to change our fitting k-path, as it is hard to fit over edges. For example we could retrieve a parabola by chosen a k-path over M$\rightarrow \Gamma \rightarrow K$. Moreover we could derive an analytical solution for the next-nearest-neighbor, or even more neighbors, tight binding method leading to more parameters improving the accuracy of our fits. This would give us the possibility to find a good set of parameters to fit both $\pi$ bands with exact accuracy. \\\\
	Within a general framework the linear energy dispersion in the vicinity of the Dirac point, leading to the loss of effective mass and relativistic electrons indicate many new phenomena. \\\\
	As the Mooresch's law is getting harder and harder to be fulfilled by silicon semiconductors, graphene is the leading hope in solving computation power problems. \\\\
	Complications are also found in the production of graphene, as in the silicon production it is very difficult to find processes of producing economic graphene lattices. Recently there have been new ideas, like growing graphene nanotubes on platinum surfaces creating a twist in the semiconductor producing.\\\\
	These called 'carbon nanotubes' are enrolled graphene layers with similar properties to graphene mono-layers. \textit{Carbon nanotubes} are used for transistors, transparent and flexible displays or even solar cells. \\\\
	Sadly an in-depth consideration of the electronic properties and transport could not be fulfilled for graphene and therefore additionally not for carbon nanotubes.\\\\
	Finally graphene will be the leading material in most of our electronic devices in 2025. Offering improvements in all electronic devices.
		