%----------------------------
% B A C H E L O R A R B E I T
%----------------------------



% DOKUMENTEINSTELLUNGEN
% ---------------------

% Enkodierung
	\usepackage[utf8]{inputenc} %Linux
	%\usepackage[latin1]{inputenc} %Windows
	\usepackage[T1]{fontenc}
	\usepackage{lmodern}
% verwendete Sprachen
	\usepackage[ngerman,english]{babel}
	
	
% ZUSÄTZLICHE PAKETE LADEN
% ------------------------

% Mathematik Pakete
	\usepackage{amsmath}
	\usepackage{amsfonts}
	\usepackage{amssymb}
	\usepackage{fixmath}


% BENUTZERDEFINIERTES AUSSEHEN
% ----------------------------

% Zeilen und Absatzabstand
	%\onehalfspacing
	\linespread{1.1}
	% einfacher Zeilenabstand in Fußnoten
	\let\footnoteOld\footnote
	\renewcommand{\footnote}[1]{\linespread{0.95}\footnoteOld{#1}}
	%\setlength{\parskip}{5pt}


% BENUTZERDEFINIERTE BEFEHLE
% --------------------------

% Mathematische Zeichen aufrecht setzen
	\newcommand{\dd}{\mathrm{d}} % Differential
	\newcommand{\ee}{\mathrm{e}} % Eulersche Zahl
	\newcommand{\ii}{\mathrm{i}} % imaginäre Einheit
% Physikalische Einheiten aufrecht setzen
	%\newcommand{\unit}[1]{\mathrm{#1}}
% Mehrzeilige Kommentare unter Klammern
	%\newcommand{\mlunderbrace}[3]{\underbrace{#2}_{\parbox{#1}{\begin{scriptsize} #3 \end{scriptsize}}}}
% Betragsstriche
	\providecommand{\abs}[1]{\left|#1\right|}
% Norm (doppelte Betragsstriche)
	%\newcommand{\norm}[1]{\left|\left|#1\right|\right|}
% "Setze" Symbol (= mit ! drüber)
	\newcommand{\set}{\stackrel{!}{=}}
% "Zu Zeigen" Symbol (ineinanderhängende ZZ)
	%\newcommand{\ZZ}{\mathrm{Z\kern-.3em\raise-0.5ex\hbox{Z}}}
% Operatoren
	\newcommand{\Var}{\mathrm{Var}}
% Vektorielle Differentialoperatoren
	%\newcommand{\vnabla}{\vec{\nabla}}
	%\newcommand{\laplace}{\Delta}
	%\newcommand{\dalembert}{\square}
% Hilbertraumvektoren
	%\newcommand{\bra}[1]{\left \langle #1 \right |}
	%\newcommand{\ket}[1]{\left | #1 \right \rangle}